\documentclass[a4paper]{article}

%%%%%%%%%%%%%%%%%%%%%%%%%%%%%%%%%%%%%%%%%%%%%%%%%%%%%%%%%%%%%%%%%%%%%%%%%%%%
% Some common includes. Add additional includes you need.
%%%%%%%%%%%%%%%%%%%%%%%%%%%%%%%%%%%%%%%%%%%%%%%%%%%%%%%%%%%%%%%%%%%%%%%%%%%%
\RequirePackage{ngerman}
\RequirePackage[utf8]{inputenc}
\RequirePackage[T1]{fontenc}
\RequirePackage[margin=23mm,bottom=30mm]{geometry}
\RequirePackage{graphicx}
\RequirePackage{amsmath,amsfonts,amssymb,amsthm}
\RequirePackage{hyperref}

%%%%%%%%%%%%%%%%%%%%%%%%%%%%%%%%%%%%%%%%%%%%%%%%%%%%%%%%%%%%%%%%%%%%%%%%%%%%
% Defines for mathematical notation. Add additional defines as needed.
%%%%%%%%%%%%%%%%%%%%%%%%%%%%%%%%%%%%%%%%%%%%%%%%%%%%%%%%%%%%%%%%%%%%%%%%%%%%
\def\O{\mathcal{O}}
\def\sort{\mathrm{sort}}
\def\scan{\mathrm{scan}}
\def\dist{\mathrm{dist}}

\setlength{\parindent}{0cm}
\renewcommand{\refname}{Quellen}
%%%%%%%%%%%%%%%%%%%%%%%%%%%%%%%%%%%%%%%%%%%%%%%%%%%%%%%%%%%%%%%%%%%%%%%%%%%%
% Definition of the assignment header
%%%%%%%%%%%%%%%%%%%%%%%%%%%%%%%%%%%%%%%%%%%%%%%%%%%%%%%%%%%%%%%%%%%%%%%%%%%%
\input{/Users/larspetersen/JWGU/Prg/header.tex}
%%%%%%%%%%%%%%%%%%%%%%%%%%%%%%%%%%%%%%%%%%%%%%%%%%%%%%%%%%%%%%%%%%%%%%%%%%%%

% Set option "german" or "english", depending on what language the
% default texts should be in.
\ExecuteOptions{german}
\ProcessOptions

% Enter the lecture name and semester
\lecture{Einf\"uhrung in die Programmierung}
\semester{Winter 2016/2017}

% Enter your data: Name, Matrikelnummer (student ID number) and group
\student{Qasim Raza, Lars Petersen}{6360278, 6290157}{11}
% Tutorin: sabrinasafre@gmail.com

% Which assignment is this?
\assignment{3}

% The environment "exercise" takes one parameter (the exercise number). 
% This way you can skip exercises if you like. Example:
% 
% \assignment{3}
% \begin{exercise}{8}
% ...
% \end{exercise}
% 
% The solution to exercise 3.8 (3rd assignment, 8th exercise) goes where 
% the dots are.


\begin{document}



\begin{exercise}{2}

\begin{itemize}

\item[(a)] Dokumentation der Testf\"alle f\"ur das einfache Programm:


\begin{center}
\begin{tabular}{| p{2.5cm} | p{2.2cm} | p{10cm} |}
		\hline
		Funktionalit\"at & Eingabe & Verhalten des Programms\\ \hline \hline
		
		Begr\"u\ss{}ung & 
		& In der Konsole wird eine allgemeine Begr\"u\ss{}ung ausgegeben. \\ \hline
		
		Reihenfolge der Spielz\"uge & 
		& Die Spieler sind abwechselnd an der Reihe. In der Konsole wird
		angezeigt, welcher Spieler den n\"achsten Zug vornehmen soll.\\ \hline
		
		Visualisierung Spielfeld & 
		& Das Spielfeld wird nach jedem Zug in aktualisierter Form in der Konsole dargestellt. \\ \hline
		
		Spaltenangabe des Spielers & Sonderzeichen \newline oder Buchstabenfolge 
		& Eingabe wird abgefangen. Der Spieler wird darauf hingewiesen,
		dass er keine ganze Zahl eingegeben hat.\\ \hline
		
		Spaltenangabe des Spielers & \texttt{int}-Literal \newline $< 1$ oder $> 8$
		& Eingabe wird abgefangen. Der Spieler wird darauf hingewiesen,
		dass die angegebene Spalte nicht im vorgegebenen Bereich lag oder die Spalte bereits vollst\"andig belegt ist. \\ \hline
		
		Spaltenangabe des Spielers & Keyboard \newline Interrupt
		& Eingabe wird abgefangen. Das Spiel wird beendet. Die Spieler erhalten in der Konsole den Hinweis, dass ein
		Keyboard Interrupt erfolgt ist.\\ \hline
		
		Spielgewinn &
		& Es wird NICHT gepr\"uft, ob ein Zug ein Gewinnzug ist oder nicht. \\ \hline
		
		Spielende \newline (regul\"ar) &
		& Nach 48 erfolgten Z\"ugen endet das Spiel, da alle Felder belegt sind und kein weiterer Zug mehr m\"oglich ist.
		Die Spieler erhalten in der Konsole den Hinweis, dass das Spiel beendet ist. \\ \hline
		\end{tabular}
\end{center}


\item[(b)] Dokumentation der Testf\"alle f\"ur das erweiterte Programm. Die ersten sechs der in (a) genannten Testf\"alle sind auch hier anwendbar und deshalb nicht mehr zus\"atzlich aufgef\"uhrt.

\begin{center}
	\begin{tabular}{| p{2.5cm} | p{2.2cm} | p{10cm} |}
		\hline
		Funktionalit\"at & Eingabe & Verhalten des Programms\\ \hline \hline
		
		Spielgewinn &
		& Nach jedem Zug eines Spielers wird gepr\"uft, ob eine der Gewinnkonfiguration erreicht wurde. Falls ja, wird das
		Spiel beendet. Die Spieler erhalten in der Konsole den Hinweis, wer gewonnen hat und dass das Spiel beendet
		ist. \\ \hline
		
		Spielgewinn & [1]: 2, 3, 4, 4 \newline [2]: 1, 1, 1 &
		Spieler 1 hat gewonnen. Das Spiel ist beendet. Beide Informationen erscheinen in der Konsole. \\ \hline
		
		Spielgewinn & [1]: 1, 1, 1, 1 \newline [2]: 2, 3, 4, 4 &
		Spieler 2 hat gewonnen. Das Spiel ist beendet. Beide Informationen erscheinen in der Konsole. \\ \hline
		
		Spielende \newline (regul\"ar) &
		& Nach 48 erfolgten Z\"ugen ohne Gewinner endet das Spiel unentschieden, da alle Felder belegt sind und kein
		weiterer Zug mehr m\"oglich ist. Die Spieler erhalten in der Konsole den Hinweis, dass das Spiel unentschieden
		endete. \\ \hline
	\end{tabular}
\end{center}

\newpage

\item[(c)] Dokumentation der Testf\"alle f\"ur das Programm, dass auch den Computer als Gegner nutzen kann.
Die ersten sechs der in (a) genannten Testf\"alle sind auch hier anwendbar und deshalb nicht mehr zus\"atzlich aufgef\"uhrt. Zudem sind die Testf\"alle aus (b) ebenfalls \"ubertragbar, so dass der Zwei-Spieler-Modus abgedeckt ist und nicht mehr gesondert aufgef\"uhrt wird.

\begin{center}
	\begin{tabular}{| p{2.5cm} | p{2.2cm} | p{10cm} |}
		\hline
		Funktionalit\"at & Eingabe & Verhalten des Programms\\ \hline \hline
		
		Wahl des \newline Spielmodus & $0$ oder $1$
		& Bei $0$ wird der Zwei\--Spieler\--Modus gew\"ahlt. Bei $1$ das Spiel gegen den Computer. Der Modus wird dem Nutzer anf\"anglich angezeigt. \\ \hline
		
		Wahl der Spielreihenfolge im Spiel gegen den Com\-puter & $0$ oder $1$
		& Bei $0$ wird der Zwei-Spieler-Modus gew\"ahlt. Bei $1$ das Spiel gegen den Computer.\\ \hline

		Spielgewinn &
		& Nach jedem Zug $-$ Spieler oder Computer $-$ wird gepr\"uft, ob eine der Gewinnkonfiguration erreicht wurde. Falls ja, wird das
		Spiel beendet. In der Konsole wird angezeigt, wer gewonnen hat und dass das Spiel beendet
		ist. \\ \hline
				
		Spielende \newline (regul\"ar) &
		& Nach 48 erfolgten Z\"ugen ohne Gewinner $-$ Mensch und Computer $-$ endet das Spiel unentschieden, da alle Felder belegt sind und kein weiterer Zug mehr m\"oglich ist. In der Konsole wird angezeigt, dass das Spiel unentschieden endete. \\ \hline
	\end{tabular}
\end{center}


\end{itemize}

\end{exercise}

%\begin{thebibliography}{99}
%\bibitem{Test}
%Hallo
%\end{thebibliography}

%\begin{center}
%\includegraphics[width=6cm,angle=270]{chart.eps}
%\end{center}

\end{document}
