\documentclass[a4paper]{article}

%%%%%%%%%%%%%%%%%%%%%%%%%%%%%%%%%%%%%%%%%%%%%%%%%%%%%%%%%%%%%%%%%%%%%%%%%%%%
% Some common includes. Add additional includes you need.
%%%%%%%%%%%%%%%%%%%%%%%%%%%%%%%%%%%%%%%%%%%%%%%%%%%%%%%%%%%%%%%%%%%%%%%%%%%%
\RequirePackage{ngerman}
\RequirePackage[utf8]{inputenc}
\RequirePackage[T1]{fontenc}
\RequirePackage[margin=23mm,bottom=30mm]{geometry}
\RequirePackage{graphicx}
\RequirePackage{amsmath,amsfonts,amssymb,amsthm}
\RequirePackage{hyperref}

%%%%%%%%%%%%%%%%%%%%%%%%%%%%%%%%%%%%%%%%%%%%%%%%%%%%%%%%%%%%%%%%%%%%%%%%%%%%
% Defines for mathematical notation. Add additional defines as needed.
%%%%%%%%%%%%%%%%%%%%%%%%%%%%%%%%%%%%%%%%%%%%%%%%%%%%%%%%%%%%%%%%%%%%%%%%%%%%
\def\O{\mathcal{O}}
\def\sort{\mathrm{sort}}
\def\scan{\mathrm{scan}}
\def\dist{\mathrm{dist}}

\setlength{\parindent}{0cm}
\renewcommand{\refname}{Quellen}
%%%%%%%%%%%%%%%%%%%%%%%%%%%%%%%%%%%%%%%%%%%%%%%%%%%%%%%%%%%%%%%%%%%%%%%%%%%%
% Definition of the assignment header
%%%%%%%%%%%%%%%%%%%%%%%%%%%%%%%%%%%%%%%%%%%%%%%%%%%%%%%%%%%%%%%%%%%%%%%%%%%%
\input{/Users/larspetersen/JWGU/Prg/header.tex}
%%%%%%%%%%%%%%%%%%%%%%%%%%%%%%%%%%%%%%%%%%%%%%%%%%%%%%%%%%%%%%%%%%%%%%%%%%%%

% Set option "german" or "english", depending on what language the
% default texts should be in.
\ExecuteOptions{german}
\ProcessOptions

% Enter the lecture name and semester
\lecture{Einf\"uhrung in die Programmierung}
\semester{Winter 2016/2017}

% Enter your data: Name, Matrikelnummer (student ID number) and group
\student{Qasim Raza, Lars Petersen}{6360278, 6290157}{11}
% Tutorin: sabrinasafre@gmail.com

% Which assignment is this?
\assignment{4}

% The environment "exercise" takes one parameter (the exercise number). 
% This way you can skip exercises if you like. Example:
% 
% \assignment{3}
% \begin{exercise}{8}
% ...
% \end{exercise}
% 
% The solution to exercise 3.8 (3rd assignment, 8th exercise) goes where 
% the dots are.


\begin{document}



\begin{exercise}{x}

Dokumentation der Testf\"alle:


\begin{center}
\begin{tabular}{| p{2.5cm} | p{2.2cm} | p{10cm} |}
		\hline
		Funktionalit\"at & Eingabe & Verhalten des Programms\\ \hline \hline
		
		Begr\"u\ss{}ung & 
		& In der Konsole wird eine allgemeine Begr\"u\ss{}ung ausgegeben. \\ \hline
		
		Startwerte randomisiert & 
		& \\ \hline
		
		Startwerte vom Spieler & 
		& \dots \\ \hline
		
		Sonderzeichen &
		& \dots \\ \hline
		
		Anzeige des Stra\ss{}ennetzes &
		& \dots \\ \hline
		
		Anzeige des Spielstatus zu Beginn eines Spieltages &
		& \dots \\ \hline
		
		Ablauf einer Spielrunde &
		& \dots \\ \hline

		Spielzug \texttt{pass} &
		& \dots \\ \hline
		
		Spielzug \texttt{move} &
		& \dots \\ \hline
		
		Spielzug \texttt{build} &
		& \dots \\ \hline
		
		Spielzug \texttt{hire} &
		& \dots \\ \hline
		
		Berechnung des Tagesgewinns &
		& Am Ende eines Spieltages \dots \\ \hline
		
		Anzeige des Spielstatus am Ende eines Tages &
		& Am Ende eines Tages \dots \\ \hline
		
		Aufruf der Hilfe &
		& Jederzeit \dots \\ \hline
	
		Beginn eines neuen Spiels &
		& Jederzeit \dots \\ \hline
		
		Spielabbruch auf Wunsch des Spielers &
		& Jederzeit \dots \\ \hline
		
		Berechnung des Gesamtgewinns &
		& Am Ende eines Spieltages \dots \\ \hline
		
		Pr\"ufung auf Highscore &
		& Am Ende des Spiels \dots \\ \hline

		Darstellung der Highscores &
		& Am Ende des Spiels \dots\\ \hline
				
		Abspeichern der Highscore-Liste &
		& Am Ende des Spiels \dots \\ \hline
		
		Regul\"ares Spielende &
		& Am Ende des Spiels \dots \\ \hline
		
	\end{tabular}
\end{center}

\end{exercise}

%\begin{thebibliography}{99}
%\bibitem{Test}
%Hallo
%\end{thebibliography}

%\begin{center}
%\includegraphics[width=6cm,angle=270]{chart.eps}
%\end{center}

\end{document}
